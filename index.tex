\documentclass[]{article}
\usepackage{lmodern}
\usepackage{amssymb,amsmath}
\usepackage{ifxetex,ifluatex}
\usepackage{fixltx2e} % provides \textsubscript
\ifnum 0\ifxetex 1\fi\ifluatex 1\fi=0 % if pdftex
  \usepackage[T1]{fontenc}
  \usepackage[utf8]{inputenc}
\else % if luatex or xelatex
  \ifxetex
    \usepackage{mathspec}
  \else
    \usepackage{fontspec}
  \fi
  \defaultfontfeatures{Ligatures=TeX,Scale=MatchLowercase}
\fi
% use upquote if available, for straight quotes in verbatim environments
\IfFileExists{upquote.sty}{\usepackage{upquote}}{}
% use microtype if available
\IfFileExists{microtype.sty}{%
\usepackage{microtype}
\UseMicrotypeSet[protrusion]{basicmath} % disable protrusion for tt fonts
}{}
\usepackage[margin=1in]{geometry}
\usepackage{hyperref}
\hypersetup{unicode=true,
            pdftitle={Welcome to NRES 746},
            pdfauthor={Advanced Analysis Methods in Natural Resources and Environmental Science},
            pdfborder={0 0 0},
            breaklinks=true}
\urlstyle{same}  % don't use monospace font for urls
\usepackage{longtable,booktabs}
\usepackage{graphicx,grffile}
\makeatletter
\def\maxwidth{\ifdim\Gin@nat@width>\linewidth\linewidth\else\Gin@nat@width\fi}
\def\maxheight{\ifdim\Gin@nat@height>\textheight\textheight\else\Gin@nat@height\fi}
\makeatother
% Scale images if necessary, so that they will not overflow the page
% margins by default, and it is still possible to overwrite the defaults
% using explicit options in \includegraphics[width, height, ...]{}
\setkeys{Gin}{width=\maxwidth,height=\maxheight,keepaspectratio}
\IfFileExists{parskip.sty}{%
\usepackage{parskip}
}{% else
\setlength{\parindent}{0pt}
\setlength{\parskip}{6pt plus 2pt minus 1pt}
}
\setlength{\emergencystretch}{3em}  % prevent overfull lines
\providecommand{\tightlist}{%
  \setlength{\itemsep}{0pt}\setlength{\parskip}{0pt}}
\setcounter{secnumdepth}{0}
% Redefines (sub)paragraphs to behave more like sections
\ifx\paragraph\undefined\else
\let\oldparagraph\paragraph
\renewcommand{\paragraph}[1]{\oldparagraph{#1}\mbox{}}
\fi
\ifx\subparagraph\undefined\else
\let\oldsubparagraph\subparagraph
\renewcommand{\subparagraph}[1]{\oldsubparagraph{#1}\mbox{}}
\fi

%%% Use protect on footnotes to avoid problems with footnotes in titles
\let\rmarkdownfootnote\footnote%
\def\footnote{\protect\rmarkdownfootnote}

%%% Change title format to be more compact
\usepackage{titling}

% Create subtitle command for use in maketitle
\newcommand{\subtitle}[1]{
  \posttitle{
    \begin{center}\large#1\end{center}
    }
}

\setlength{\droptitle}{-2em}
  \title{Welcome to NRES 746}
  \pretitle{\vspace{\droptitle}\centering\huge}
  \posttitle{\par}
  \author{Advanced Analysis Methods in Natural Resources and Environmental Science}
  \preauthor{\centering\large\emph}
  \postauthor{\par}
  \predate{\centering\large\emph}
  \postdate{\par}
  \date{Fall 2018}


\begin{document}
\maketitle

\subsubsection{Instructor}\label{instructor}

Kevin Shoemaker\\
Office: FA 220E\\
Phone: (775)682-7449\\
Email: \url{kshoemaker@cabnr.unr.edu}\\
Office hours: M and F from 11 to noon in FA 220E

\subsubsection{Course Meeting Times}\label{course-meeting-times}

\textbf{Lecture \& Discussion}: M, W at 10am in FA 337 (50 mins)\\
\textbf{Lab}: Tuesday at 3pm in LME 315 (3 hours)

\subsubsection{Course Website}\label{course-website}

\href{http://naes.unr.edu/shoemaker/teaching/NRES-746/index.html}{http://naes.unr.edu/shoemaker/teaching/NRES-746/}

\subsubsection{Course Objectives}\label{course-objectives}

Modern computers have reduced or eliminated many of the barriers to
advanced data analysis, and as a result computational algorithms now
often have primacy over elegant and simple mathematical formulae for
complex data analysis. Armed with basic concepts of probability and
statistics, and with some facility with computer programming, ecologists
and natural resource professionals get more out of their data than ever
before. \textbf{In this course, we embrace the primacy of the
algorithm}.

By the end of this course, students should have the ability to (1)
develop computational routines to simulate data generation under
alternative mechanisms, (2) fit these computational models to data using
maximum likelihood and Bayesian inference, (3) validate these
computational models, and (4) understand where and when to use a wide
variety of additional advanced data analysis methods. The goal is for
students to emerge from this course as creative data analysts with the
tools and intuition needed to draw inferences from a wide variety of
data types.

\textbf{The course motto}: \emph{Be Dangerous}! What does that mean?? It
is \emph{safer} to use standard analytical tools (e.g., in a software
like SAS or SPSS) because these methods have been validated and tested
in many ways over the years. When we build our own algorithms, we can be
entering uncharted territory. And exploring these territories can be
dangerous\ldots{} And daunting\ldots{} And exciting!

The focus of this course is on using computational algorithms to infer
ecological processes and relationships from pattern in
\emph{observational} studies; \emph{we will not directly address
experimental methods or design in this course}. However, the data
analysis methods covered will be of broad utility for a wide variety of
disciplines and data types. The general focus will be on predictive
statistical modeling methods, including regression-based approaches,
hierarchical/mixed models, and multi-model inference. Additional
student-led modules will cover other advanced analysis topics such as
classification and regression trees, structural equation modeling, and
geographic models of species distributions. \emph{This is not a
``statistics'' course per se}; we will focus on the implementation and
leave the nitty-gritty stats questions to statisticians.

Each student will be responsible for leading discussions and
demonstrations on a data analysis method of their choice (working in
groups). The laboratory portion of the class will provide students the
opportunity to try out some of the data analysis methods. Structured
labs with example data sets will be interspersed with open lab periods
where students work in small groups on a research project involving
analysis of a real-world data set.

\subsubsection{Student Learning
Objectives}\label{student-learning-objectives}

\begin{enumerate}
\def\labelenumi{\arabic{enumi}.}
\tightlist
\item
  Identify and contrast the major classes of statistical models used by
  ecologists (e.g., Bayesian vs frequentist, likelihood-based, machine
  learning) and explain how and why ecologists use these models.
\item
  Apply analysis tools such as logistic regression, non-linear
  regression models, mixed-effects models, and machine-learning methods
  (e.g., Random Forest) on diverse data sets representative of those
  commonly considered in observational studies in ecology.
\item
  Learn to explore data sets quantitatively and graphically and to
  prepare data appropriately for analysis.
\item
  Perform statistical analysis, data visualization, simulation modeling,
  model validation and programming with the statistical computing
  language R.
\item
  Critically evaluate the strength of inferences drawn from statistical
  models by understanding and testing major assumptions and using tools
  such as cross-validation.
\item
  Communicate statistical and computational concepts by leading lectures
  and discussion on advanced topics in data analysis.
\end{enumerate}

\subsubsection{Prerequisites}\label{prerequisites}

Curious scientific mind, broad research interests, comfort with (or at
least, lack of fear regarding) quantitative topics. Students are
expected to already have a fundamental knowledge of relevant statistical
methods, obtained through other coursework. If this is not the case,
they should be prepared to work harder to develop the necessary
prerequisite knowledge.

\subsubsection{Textbooks and Readings}\label{textbooks-and-readings}

We will use the book,
\href{https://ms.mcmaster.ca/~bolker/emdbook/}{Ecological Models and
Data in R}, by Ben Bolker, as a general class reference. However,
additional readings will be assigned, and will be available on the
course website.

In addition, readings will be assigned as indicated in the course
schedule (which is still evolving!).

\subsubsection{Grading}\label{grading}

\begin{longtable}[]{@{}ll@{}}
\toprule
Course component & Weight\tabularnewline
\midrule
\endhead
Student-led topics & 20\%\tabularnewline
Participation & 20\%\tabularnewline
Laboratories & 20\%\tabularnewline
Research Project, written & 30\%\tabularnewline
Research Project, presentation & 10\%\tabularnewline
\bottomrule
\end{longtable}

\subsubsection{Course components}\label{course-components}

\textbf{Student-led presentations}: Each student will be responsible for
leading a lecture/discussion that introduces a data analysis method
using a worked example (clear, concise, informative tutorial), and a
discussion of applications from the published literature. Presenters are
encouraged to work with the instructor (and other faculty, graduate
students!) to better understand their data, methods, papers and
topics.\\
\textbf{Class Participation}: Students are expected to actively
participate in the classroom education process. Don't be afraid to ask
questions- fear of embarrassment can be a major impediment to learning.
So consider this a safe space for making mistakes- \emph{be dangerous
AND make a fool of yourself!}\\
\textbf{Laboratory Reports}: Students will submit (1) R functions that
accomplish specific tasks (that will be graded using an automated
algorithm- in R of course!), and (2) a brief report (submitted via a
WebCampus discussion) succinctly answering any questions posed, and
stating any questions or points of confusion. While students are
encouraged to work on the labs in small groups, all lab submissions will
be made individually.\\
\textbf{Group Projects}: Students will work on projects in groups of 2 -
3. Projects will require analysis of previously published or publicly
available data sets that are NOT intended to be part of a student's
planned thesis or dissertation chapters. The instructor can assist with
identifying suitable data sets. Although a primary goal is to enhance
knowledge and facility with the data analysis methods, an important
secondary goal could be to develop a collaborative manuscript for
publication! Therefore, careful thought should go into choice of a data
set and relevant scientific questions to guide the analysis. The group
project will take the form of a manuscript suitable for submission as a
research paper. This will be submitted to the instructor as a complete
draft by Dec 8, 2018, and (after review and comment by the instructor)
as a final version by December 19.

\paragraph{Group projects:
expectations}\label{group-projects-expectations}

Students are expected to perform (and write up the results for) a data
analysis using \emph{state-of-the-art analytical methods}. The write-up
will loosely take the format of a scientific paper to be submitted to a
professional journal. However, because of the nature of this course, the
most important pieces of the write-up are the \textbf{methods} and
\textbf{results} sections. Nonetheless, I expect at least a few
paragraphs introducing the topic and why it's important, and a few
paragraphs discussing the implications of the results. The methods and
results section can (and in many cases should) be much longer than you
typically see in a scientific paper- don't feel constrained by space for
these sections! Not that you need to be wordy, I just want to make sure
you have the space to clearly explain the analyses you performed and why
you made the choices you did!

Here is a more detailed description of expectations for the final group
project:

\textbf{Introduction:} Provide enough description so that the reader
understands why the research is important and (if appropriate) what
question(s) are being addressed.

\textbf{Methods:}\\
- Provide just enough details about the data collection to give the
reader the context necessary to understand the data.\\
- Provide plenty of detail about the analytical approach- enough detail
to fully replicate the analysis. Justify all decisions that were made
and (where appropriate) discuss why you did not use alternative
approaches.\\
- Discuss key analytical assumptions.\\
NOTE: This section can be longer than the methods section of a standard
manuscript.

\textbf{Results:} Present all relevant results completely and concisely.
Wherever possible, results should be presented via figures and tables.
There is a limit of 5 figures and 3 tables, so choose carefully which
figures and tables to present. Figures should be \emph{publication
quality}.

\textbf{Discussion:} Write at least three paragraphs that put the
results in a larger context (returning to the key research questions)
and discuss areas of uncertainty. Potential topics are possible
violations of assumptions, and future work that your analysis suggests
would be profitable.\\
\textbf{Code} Provide all code used to run the analyses presented in the
paper as an R script.

\paragraph{Group project overview (1 page, due
9/25)}\label{group-project-overview-1-page-due-925}

Please provide a 1-page project description. Make sure you include:

\begin{itemize}
\tightlist
\item
  Project title
\item
  Participant names
\item
  Participant roles (brief description of how you plan to divide the
  tasks up)
\item
  Brief background (motivation for research question)
\item
  Research question(s)\\
\item
  Primary and ancillary data sources
\item
  Anticipated analytical approach
\end{itemize}

\subsubsection{\texorpdfstring{\href{schedule.html}{Course
Schedule}}{Course Schedule}}\label{course-schedule}

NOTE: the course schedule is not set in stone, so please check back
frequently!

\href{schedule.html}{http://naes.unr.edu/shoemaker/teaching/NRES-746/schedule.html}

\subsubsection{\texorpdfstring{\href{labschedule.html}{Lab
Schedule}}{Lab Schedule}}\label{lab-schedule}

\href{labschedule.html}{http://naes.unr.edu/shoemaker/teaching/NRES-746/labschedule.html}

\subsubsection{Make-up policy and late
work:}\label{make-up-policy-and-late-work}

If you miss a class meeting or lab period, it is your responsibility to
talk to one of your classmates about what you missed. If you miss a lab
meeting, you are still responsible for completing the lab activities and
write-up on your own time. You do not need to let me know in advance
that you are going to miss class or lab.

\subsubsection{Students with
Disabilities}\label{students-with-disabilities}

Any student with a disability needing academic adjustments or
accommodations is requested to speak with the Disability Resource Center
(Thompson Building, Suite 101) as soon as possible to arrange for
appropriate accommodations.

\subsubsection{Statement on Academic
Dishonesty}\label{statement-on-academic-dishonesty}

Cheating, plagiarism or otherwise obtaining grades under false pretenses
constitute academic dishonesty according to the code of this university.
Plagiarism is using the ideas or words of another person without giving
credit to the original source; this includes copying another student in
class. Always cite the source of your information. This includes copying
or paraphrasing from a book, journal, or unpublished material without
giving credit to the author(s), and submitting a term paper that was
used in another course. Academic dishonesty will not be tolerated and
penalties can include filing a final grade of ``F''; reducing the
student's final course grade one or two full grade points; awarding a
failing mark on the coursework in question; or requiring the student to
retake or resubmit the coursework. For more details, see the
\href{http://catalog.unr.edu/}{University of Nevada, Reno General
Catalog}.

\subsubsection{This is a safe space}\label{this-is-a-safe-space}

The University of Nevada, Reno is committed to providing a safe learning
and work environment for all. If you believe you have experienced
discrimination, sexual harassment, sexual assault, domestic/dating
violence, or stalking, whether on or off campus, or need information
related to immigration concerns, please contact the University's Equal
Opportunity \& Title IX Office at 775-784-1547. Resources and interim
measures are available to assist you. For more information, please
visit: \url{http://www.unr.edu/equal-opportunity-title-ix}"

\subsubsection{Statement on Audio and Video
Recording}\label{statement-on-audio-and-video-recording}

Surreptitious or covert video-taping of class or unauthorized audio
recording of class is prohibited by law and by Board of Regents policy.
This class may be videotaped or audio recorded only with the written
permission of the instructor. In order to accommodate students with
disabilities, some students may have been given permission to record
class lectures and discussions. Therefore, students should understand
that their comments during class may be recorded.


\end{document}
